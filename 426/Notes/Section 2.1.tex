\documentclass[11pt]{article}

\usepackage{exscale}
\usepackage{graphicx}
\usepackage{amsmath}
\usepackage{latexsym}
\usepackage{times,mathptm}
\usepackage{epsfig}

\textwidth 6.5truein          
\textheight 9.0truein
\oddsidemargin 0.0in
\topmargin -0.6in

\parindent 0pt          
\parskip 5pt
\def\baselinestretch{1.1}

\begin{document}
\centerline{VJ Davey}\vskip 0.10cm
\begin{LARGE}
\centerline {\bf CSci 426 Notes Section 2.1}
\end{LARGE}
\vskip 0.25cm
\centerline{\bf Lehmer Random Number Generation}
\begin{enumerate}
	\item \textbf{Random Number Generation}
		\\Algorithmic Generators satisfy the well accepted Random Number Generation criteria:
		\\\textit{Randomness...Controllability...Portability...Efficiency...Documentation}
		\\An ideal generator is a function that produces a real number between 0 and 1 where each number has an equal chance of being selected.
		\\We would want to be able to choose any number m in a set $x_m = \{1,2,3...m\}$.
	\item \textbf{Lehmer's Algorithm}
			\\Is defined in terms of two fixed parameters:
				\begin{enumerate}
					\\\textit{modulus m}, a fixed large prime integer.
					\\\textit{multiplier a}, a fixed integer in $ x_m $.
				\end{enumerate}
			\\And the subsequent generation of the integer sequence $\{x_0, x_1, x_2...x_m\}$ via the iterative equation $$x_{i+1} = g(x_i)$$ where $g(x)$ is defined for all $x\in x_m$ as $$g(x) = ax \mod{m}$$
			\\ The initial seed $x_0$ is chosen from the set $x_m$.
		\\The modulus operation always causes the remainder to fall between 0 and $m-1$.
		\\If 0 is used as a seed, all subsequent values of the sequence will be 0. So, $g(x) \neq 0$ for any  $x \in x_m$.
		\\$g: x_m \rightarrow x_m$
		\\There is nothing actually random about a random number generator. 
		\\\textit{When choosing $(a,m)$ we need the function to generate a full period sequence, and we need the sequence to appear to be random.}
		\vskip 5.5cm
		\\\textbf{Full Period Multipliers}
		\\A multiplier is \textbf{full period} iff. the fundamental period p  of the sequence produced is m-1. Or, more simply, it contains all numbers between 1 and m-1 as distinct elements of a list.
		\\The following algorithm can be used to determine whether a multiplier a is full period relative to the prime modulus m:
		\begin{verbatim}
			p = 1;
			x = a;
			while (x != 1) {
			    p++;
			    x = (a * x) % m;
			}
			if (p == m - 1) {
		        //a is a full period multiplier
			}else{
			    //a is not a full period multiplier
			}
		\end{verbatim}
\end{enumerate}
\end{document}